\ifx\allfiles\undefined
\documentclass[12pt, a4paper, oneside, UTF8]{ctexbook}
\def\path{../config}
\usepackage{amsmath}
\usepackage{amsthm}
\usepackage{amssymb}
\usepackage{graphicx}
\usepackage{mathrsfs}
\usepackage{enumitem}
\usepackage{geometry}
\usepackage[colorlinks, linkcolor=black]{hyperref}
\usepackage{stackengine}
\usepackage{yhmath}
\usepackage{extarrows}

\usepackage{multicol}
\usepackage{fancyhdr}
\usepackage[dvipsnames, svgnames]{xcolor}
\usepackage{listings}
\usepackage{subfigure}
\usepackage{tikz}


\definecolor{mygreen}{rgb}{0,0.6,0}
\definecolor{mygray}{rgb}{0.5,0.5,0.5}
\definecolor{mymauve}{rgb}{0.58,0,0.82}

\graphicspath{ {figure/},{../figure/}, {config/}, {../config/} }

\linespread{1.6}

\geometry{
    top=25.4mm, 
    bottom=25.4mm, 
    left=20mm, 
    right=20mm, 
    headheight=2.17cm, 
    headsep=4mm, 
    footskip=12mm
}

\setenumerate[1]{itemsep=5pt,partopsep=0pt,parsep=\parskip,topsep=5pt}
\setitemize[1]{itemsep=5pt,partopsep=0pt,parsep=\parskip,topsep=5pt}
\setdescription{leftmargin=4em,itemsep=5pt,partopsep=0pt,parsep=\parskip,topsep=5pt}

\lstset{
    language=Mathematica,
    basicstyle=\tt,
    breaklines=true,
    keywordstyle=\bfseries\color{NavyBlue}, 
    emphstyle=\bfseries\color{Rhodamine},
    commentstyle=\itshape\color{black!50!white}, 
    stringstyle=\bfseries\color{PineGreen!90!black},
    columns=flexible,
    numbers=left,
    numberstyle=\footnotesize,
    frame=tb,
    breakatwhitespace=false,
} 
% 定理环境
\usepackage{tcolorbox}
\tcbuselibrary{most}
\theoremstyle{definition}


\newtheorem{proposition}{\indent 命题}[section]
\newtheorem{example}{\indent \color{SeaGreen}{例}}[section]
\theoremstyle{plain}
\newtheorem*{rmk}{\indent 注}
\renewenvironment{proof}{\indent\textcolor{SkyBlue}{\textbf{证明.}}\;}{\qed\par}
\newenvironment{solution}{\indent\textcolor{SkyBlue}{\textbf{解.}}\;}{\qed\par}
% #### 将 config.tex 中的定理环境的对应部分替换为如下内容
% 定义单独编号,其他四个共用一个编号计数 这里只列举了五种,其他可类似定义(未定义的使用原来的也可)
\newtcbtheorem[number within=section]{defn}%
{定义}{colback=OliveGreen!10,colframe=Green!70,fonttitle=\bfseries}{def}

\newtcbtheorem[number within=section]{lemma}%
{引理}{colback=Salmon!20,colframe=Salmon!90!Black,fonttitle=\bfseries}{lem}

% 使用另一个计数器 use counter from=lemma
\newtcbtheorem[use counter from=lemma, number within=section]{them}%
{定理}{colback=SeaGreen!10!CornflowerBlue!10,colframe=RoyalPurple!55!Aquamarine!100!,fonttitle=\bfseries}{them}

\newtcbtheorem[use counter from=lemma, number within=section]{criterion}%
{准则}{colback=green!5,colframe=green!35!black,fonttitle=\bfseries}{cri}

\newtcbtheorem[use counter from=lemma, number within=section]{corollary}%
{推论}{colback=Emerald!10,colframe=cyan!40!black,fonttitle=\bfseries}{cor}
% colback=red!5,colframe=red!75!black

% 这个颜色我不喜欢
%\newtcbtheorem[number within=section]{proposition}%
%{命题}{colback=red!5,colframe=red!75!black,fonttitle=\bfseries}{cor}

% .... 命题 例 注 证明 解 使用之前的就可以(全文都是这种框框就很丑了),也可以按照上述定义 ...
\def\d{\mathrm{d}}
\def\R{\mathbb{R}}
\newcommand{\bs}[1]{\boldsymbol{#1}}
\newcommand{\ora}[1]{\overrightarrow{#1}}
\newcommand{\myspace}[1]{\par\vspace{#1\baselineskip}}
\newcommand{\xrowht}[2][0]{\addstackgap[.5\dimexpr#2\relax]{\vphantom{#1}}}
\newenvironment{ca}[1][1]{\linespread{#1} \selectfont \begin{cases}}{\end{cases}}
\newenvironment{vx}[1][1]{\linespread{#1} \selectfont \begin{vmatrix}}{\end{vmatrix}}
\newcommand{\tabincell}[2]{\begin{tabular}{@{}#1@{}}#2\end{tabular}}
\newcommand{\pll}{\kern 0.56em/\kern -0.8em /\kern 0.56em}
\newcommand{\dive}[1][F]{\mathrm{div}\;\bs{#1}}
\newcommand{\rotn}[1][A]{\mathrm{rot}\;\bs{#1}}

\def\myIndex{0}
% \input{\path/cover_package_\myIndex.tex}

\def\myTitle{矩阵理论复习笔记}
\def\myAuthor{}
\def\myDateCover{}
\def\myDateForeword{\\\today}
\def\myForeword{前言}
\def\myForewordText{\par
本复习笔记是我个人在学习矩阵理论的过程中整理、总结而成,包含课本的内容、上课PPT涉及到的内容以及不懂地方的补充知识,希望能够对你有所帮助。\par 全书排版是利用\LaTeX 完成的,这也是对我使用\LaTeX 的一次较大工程的练手,希望我能在撰写完之后对于\LaTeX 使用有更深层次的理解。\par 因本人水平有限,故本总结笔记如有不当之处,敬请指出,本人不胜感激!
}
\def\mySubheading{}


\begin{document}

\else
\fi
\chapter{线性代数基础}
本章将会正式进入矩阵理论的知识内容中,首先是第一章,这一章的名字叫做“线性代数基础”,按照课本的话来讲,这一部分不是单纯的对于线性代数知识的简单回顾与复习,而是在已经掌握线性代数的知识的基础上进行深化,同时,这一章也是后面内容的基础。

本章将会涉及到以下内容:
\begin{itemize}[leftmargin=4em]
    \item 线性空间与子空间
    \item 空间分解与维数定理
    \item 特征值与特征向量
    \item 线性变换
    \item \dots
\end{itemize}
\noindent
\textbf{注意}:之后的内容会随着课程的进行进行及时更新
\newpage
\section{线性空间与子空间}
提到“空间”一词,很多人对这个概念应该并不陌生,我们从出生便降临在这个世界中,这个世界便是一个三维空间,我们使用计算机浏览互联网,这也可以称作一个网络空间......类似的例子还有很多很多。

\subsection{线性空间}
在这里我们要讨论的概念叫“线性空间”,是数学上的空间,上面提到的我们生活在的三维空间,抽象出来也属于线性空间。

线性空间的例子,除了上面提到的三维空间,在平面直角坐标系$x,y$组成的一个平面也是一个空间(二维空间),那究竟何为线性空间呢?有什么标准来判断其是否是一个空间?

判断能否组成一个空间的定义如下:
\begin{defn}{判断空间的定义}{def1.1.1}
    设$V$是一个非空集合,$P$是一个数域,在集合$V$的元素之间定义加法$v=\alpha+\beta$,定义数量乘法$\delta=k\alpha$,如果加法与数量乘法满足下列规则:
    \begin{multicols}{2}
        \begin{itemize}
            \item $\alpha+\beta=\beta+\alpha$
            \item $(\alpha+\beta)+\gamma=\alpha+(\beta+\gamma)$
            \item $\exists \bs{0}\in V, \forall \alpha\in V, \text{有}\alpha+\bs{0}=\alpha$(存在零元素)
            \item $\forall\alpha\in V, \exists\beta\in V, \text{使得}\alpha+\beta=\bs{0}$(存在负元素)\\
            \item $1\alpha=\alpha$
            \item $k(l\alpha)=(kl)\alpha$
            \item $(k+l)\alpha=k\alpha+l\alpha$
            \item $k(\alpha+\beta)=k\alpha+k\beta$
        \end{itemize}
    \end{multicols}
    则$V$称为数域$P$上的\textbf{线性空间}
\end{defn}

在这里我们需要明确下面一个概念:何为数域?
\begin{defn}{数域的概念}{}
    如果一个由数字构成的集合(叫做数集)$P$,这个数集对于加法、减法、乘法、除法(除数不为0)封闭,则就把这个数集$P$叫做数域
\end{defn}

这里又引出了一个新的词——封闭,何为封闭?封闭的概念较为简单:如果一个集合中的某两个数做某一运算之后的结果仍然在该集合中,那么就称该集合对于该运算是封闭的。

如果还是对于这一概念不理解,希望下面这个例子能够帮你理解:
\begin{example}
全体整数组成的集合$\mathbb{Z}$是否是一个数域?
\end{example}
\begin{solution}
    整数集包含两大部分:\textbf{正整数和负整数},0是整数,但0既不是正数也不是负数

    接下来我们来判断整数集是否对于加法封闭:

    我们从小学数学的知识就可以得知,两个整数相加依旧是整数,所以整数集对于加法是封闭的

    依次类推,两个整数相减,相乘,结果依旧是一个整数,所以很显然,整数集对于四则运算中的加法、减法和乘法都是封闭的

    最后,整数集对于除法是否是封闭的呢?

    很显然不是,举一个最简单的例子,$a=1, b=2$,$a$除以$b$的结果是$\frac{1}{2}$,它并不是一个整数,而是一个分数,或者说小数,又可以说是一个有理数,所以整数集对于除法并不是封闭的。

    综上,可以断定,整数集并不是一个数域。
\end{solution}

\begin{rmk}
    定义\ref{def:def1.1.1}中的八条性质说明了什么?

    左侧四条定义了空间对于加法需要满足以下特性:加法的交换律、加法的结合律、存在零元素、存在父元素

    右侧的四条定义了空间的数量乘法需要满足以下特性:数量乘法的结合律、数量乘法的分配律(分配律分为两个标量相加的分配律,以及两个向量相加的分配律)。
\end{rmk}

通常,定义\ref{def:def1.1.1}给出的八个条件即为判断一个集合是否能构成空间的依据,请看下面的例题。

\begin{example}
    设多项式集合\[P_n[x]=\{a_{n-1} x^{n-1}+\cdots+a_1x+a_0 | a_i\in P, i=0,1,\cdots,n-1\} \]
    这里$P_n$代表\textbf{次数不超过$n$的多项式}, 请问$P_n[x]$是否能构成一个线性空间?
\end{example}
\begin{solution}
    按照定义\ref{def:def1.1.1}的八条规律,依次来判断

    首先,多项式的加法一定满足交换律和结合律(由小学数学学过的加法交换律和加法结合律就能得知),同样的,我们可以在这个多项式集合中找到一个元素0,使得该多项式与0相加的结果依旧是该多项式(很明显,这个元素0就是数字0,即$a_{n-1}, a_{n-2},\cdots,a_1, a_0$均为0的时候),此外,我们可以构造出下面的一个多项式集合,令其与$P_n[x]$相加的结果为0:
    \[Q_n[x]=-P_n[x]\]

    综上,该集合满足空间定义中对于加法的规律,接下来判断乘法

    很显然,存在一个元素1,使得该集合与1相乘的结果就是其本身(这个元素1就是数字1,即$a_{n-1}=a_{n-2}=\cdots=a_1=0, a_0=1$的时候)。同时,由小学数学和初中数学的知识可以知道,多项式的乘法满足数乘的结合律与分配律
    
    综上,该多项式集合是一个线性空间。
\end{solution}
\begin{rmk}
    上面的例子大多数情况下运用了一些“显然,由\text{\dots\dots}的知识可以得知”,没有具体写明如何得出的结果,有以下两点原因,第一点原因是在写这段文字的时候确实懒得打这么长的公式了,二是认为大家应该能明白上面判断的过程,所以就没有写明公式,当然写出公式也是可以的,如果后面确实不理解上面是如何“显然”得来的,会重新更新这部分的例子,用公式表明。
\end{rmk}

\subsection{线性空间的维数}
在了解了何为线性空间之后,接下来我们再来了解一下如何形容一个线性空间,维数便是形容线性空间的一个度量,我们前面一直所说的“三维空间”中的“三维”,就表明该空间的维数为3
\begin{defn}{线性空间维数的定义}{}
    在线性空间$V$中,如过有$n$个向量$\varepsilon_1, \varepsilon_2,\cdots, \varepsilon_n$线性无关,而$V$中任意$n+1$个向量线性相关,则称$\varepsilon_1, \varepsilon_2,\cdots, \varepsilon_n$为$V$的一组\textbf{基底},由于线性空间的所有基底总含有相同数目的向量,则$n$称为线性空间$V$的\textbf{维数},常记为$\dim V=n$
\end{defn}

上面的定义说明了,一个空间中线性无关向量的个数其实就是该空间的维数,同时,这一系列线性无关的向量便可以构成该空间的一组基,回顾第零章线性表示相关的内容,我们同样可以得知:该空间的任意向量都可以由这一组基来线性表示。
\newpage
\begin{rmk}
    请注意,向量组/线性空间的维数与向量的维数是两个不同的概念,一定要注意区分。

    向量的维数:向量有几行,一般就说是向量的维数为几,如$\bs{\alpha}=[1,2,3]^T$,那么这就是一个三维向量

    向量组/线性空间的维数:向量组中线性无关向量的个数,如$A=\begin{bmatrix}
        1&1&1\\
        2&2&2\\
        3&3&3
    \end{bmatrix}$,虽然$A$是一个$3\times 3$的矩阵,但是经过分析可以看出该矩阵其实只有一个线性无关的向量,故该向量组/空间是1维的,但是其中按列分块出的的三个列向量却是三维的向量。
\end{rmk}

\section{空间分解与维数定理}
//TODO
\ifx\allfiles\undefined
\end{document}
\fi