\ifx\allfiles\undefined
\documentclass[12pt, a4paper, oneside, UTF8]{ctexbook}
\def\path{../config}
\usepackage{amsmath}
\usepackage{amsthm}
\usepackage{amssymb}
\usepackage{graphicx}
\usepackage{mathrsfs}
\usepackage{enumitem}
\usepackage{geometry}
\usepackage[colorlinks, linkcolor=black]{hyperref}
\usepackage{stackengine}
\usepackage{yhmath}
\usepackage{extarrows}

\usepackage{multicol}
\usepackage{fancyhdr}
\usepackage[dvipsnames, svgnames]{xcolor}
\usepackage{listings}
\usepackage{subfigure}
\usepackage{tikz}


\definecolor{mygreen}{rgb}{0,0.6,0}
\definecolor{mygray}{rgb}{0.5,0.5,0.5}
\definecolor{mymauve}{rgb}{0.58,0,0.82}

\graphicspath{ {figure/},{../figure/}, {config/}, {../config/} }

\linespread{1.6}

\geometry{
    top=25.4mm, 
    bottom=25.4mm, 
    left=20mm, 
    right=20mm, 
    headheight=2.17cm, 
    headsep=4mm, 
    footskip=12mm
}

\setenumerate[1]{itemsep=5pt,partopsep=0pt,parsep=\parskip,topsep=5pt}
\setitemize[1]{itemsep=5pt,partopsep=0pt,parsep=\parskip,topsep=5pt}
\setdescription{leftmargin=4em,itemsep=5pt,partopsep=0pt,parsep=\parskip,topsep=5pt}

\lstset{
    language=Mathematica,
    basicstyle=\tt,
    breaklines=true,
    keywordstyle=\bfseries\color{NavyBlue}, 
    emphstyle=\bfseries\color{Rhodamine},
    commentstyle=\itshape\color{black!50!white}, 
    stringstyle=\bfseries\color{PineGreen!90!black},
    columns=flexible,
    numbers=left,
    numberstyle=\footnotesize,
    frame=tb,
    breakatwhitespace=false,
} 
% 定理环境
\usepackage{tcolorbox}
\tcbuselibrary{most}
\theoremstyle{definition}


\newtheorem{proposition}{\indent 命题}[section]
\newtheorem{example}{\indent \color{SeaGreen}{例}}[section]
\theoremstyle{plain}
\newtheorem*{rmk}{\indent 注}
\renewenvironment{proof}{\indent\textcolor{SkyBlue}{\textbf{证明.}}\;}{\qed\par}
\newenvironment{solution}{\indent\textcolor{SkyBlue}{\textbf{解.}}\;}{\qed\par}
% #### 将 config.tex 中的定理环境的对应部分替换为如下内容
% 定义单独编号,其他四个共用一个编号计数 这里只列举了五种,其他可类似定义(未定义的使用原来的也可)
\newtcbtheorem[number within=section]{defn}%
{定义}{colback=OliveGreen!10,colframe=Green!70,fonttitle=\bfseries}{def}

\newtcbtheorem[number within=section]{lemma}%
{引理}{colback=Salmon!20,colframe=Salmon!90!Black,fonttitle=\bfseries}{lem}

% 使用另一个计数器 use counter from=lemma
\newtcbtheorem[use counter from=lemma, number within=section]{them}%
{定理}{colback=SeaGreen!10!CornflowerBlue!10,colframe=RoyalPurple!55!Aquamarine!100!,fonttitle=\bfseries}{them}

\newtcbtheorem[use counter from=lemma, number within=section]{criterion}%
{准则}{colback=green!5,colframe=green!35!black,fonttitle=\bfseries}{cri}

\newtcbtheorem[use counter from=lemma, number within=section]{corollary}%
{推论}{colback=Emerald!10,colframe=cyan!40!black,fonttitle=\bfseries}{cor}
% colback=red!5,colframe=red!75!black

% 这个颜色我不喜欢
%\newtcbtheorem[number within=section]{proposition}%
%{命题}{colback=red!5,colframe=red!75!black,fonttitle=\bfseries}{cor}

% .... 命题 例 注 证明 解 使用之前的就可以(全文都是这种框框就很丑了),也可以按照上述定义 ...
\def\d{\mathrm{d}}
\def\R{\mathbb{R}}
\def\C{\mathbb{C}}
\def\a{\bs{a}}
\def\b{\bs{b}}
\def\x{\bs{x}}
\def\y{\bs{y}}
\def\z{\bs{z}}
\def\u{\bs{u}}
\def\A{\bs{A}}
\def\B{\bs{B}}
\def\D{\bs{D}}
\def\G{\bs{G}}
\def\H{\bs{H}}
\def\L{\bs{L}}
\def\Q{\bs{Q}}
\def\X{\bs{X}}
\def\Y{\bs{Y}}
\def\Z{\bs{Z}}
\def\U{\bs{U}}
\def\V{\bs{V}}
\def\P{\bs{P}}
\def\J{\bs{J}}
\def\I{\bs{I}}
\def\E{\bs{E}}
\newcommand{\bs}[1]{\boldsymbol{#1}}
\newcommand{\ora}[1]{\overrightarrow{#1}}
\newcommand{\myspace}[1]{\par\vspace{#1\baselineskip}}
\newcommand{\xrowht}[2][0]{\addstackgap[.5\dimexpr#2\relax]{\vphantom{#1}}}
\newenvironment{ca}[1][1]{\linespread{#1} \selectfont \begin{cases}}{\end{cases}}
\newenvironment{vx}[1][1]{\linespread{#1} \selectfont \begin{vmatrix}}{\end{vmatrix}}
\newcommand{\tabincell}[2]{\begin{tabular}{@{}#1@{}}#2\end{tabular}}
\newcommand{\pll}{\kern 0.56em/\kern -0.8em /\kern 0.56em}
\newcommand{\dive}[1][F]{\mathrm{div}\;\bs{#1}}
\newcommand{\rotn}[1][A]{\mathrm{rot}\;\bs{#1}}
\newcommand{\rank}{\text{rank}}

\def\myIndex{0}
% \input{\path/cover_package_\myIndex.tex}

\def\myTitle{矩阵理论复习笔记}
\def\myAuthor{}
\def\myDateCover{}
\def\myDateForeword{\\\today}
\def\myForeword{前言}
\def\myForewordText{\par
本复习笔记是我个人在学习矩阵理论的过程中整理、总结而成,包含课本的内容、上课PPT涉及到的内容以及不懂地方的补充知识,希望能够对你有所帮助。\par 全书排版是利用\LaTeX 完成的,这也是对我使用\LaTeX 的一次较大工程的练手,希望我能在撰写完之后对于\LaTeX 使用有更深层次的理解。\par 因本人水平有限,故本总结笔记如有不当之处,敬请指出,本人不胜感激!
}
\def\mySubheading{}


\begin{document}

\else
\fi
\chapter{矩阵分析}
在学习微积分的课程中,我们对于函数研究过其极限、求导、求积分等操作,能否将这些数学工具应用到矩阵中呢?这就是本章矩阵分析要干的事情。

本章内容包括:
\begin{itemize}[leftmargin=4em]
    \item 矩阵序列与矩阵级数
    \item 矩阵函数
\end{itemize}

\section{矩阵序列与矩阵级数}
\subsection{矩阵序列}
\subsubsection{从向量序列到矩阵序列}
在第二章介绍向量范数的应用的时候,我们曾经接触过向量序列的敛散性问题:\[\lim\limits_{k\to\infty}x^{k}=\lim\limits_{k\to\infty}\Vert x^{k}-a\Vert=0\]
也在相对应的地方介绍了何为向量序列,向量序列也是通过数列的情况延伸而来,再对向量序列进行延伸,便有了矩阵序列的概念
\begin{defn}{矩阵序列}{}
    设$m\times n$型矩阵序列为$\A^{(k)}$,其中\[\A^{(k)}=\begin{bmatrix}
        a_{11}^{(k)}&a_{12}^{(k)}&\cdots&a_{1n}^{(k)}\\
        a_{21}^{(k)}&a_{22}^{(k)}&\cdots&a_{2n}^{(k)}\\
        \vdots&\vdots&\vdots&\vdots\\
        a_{m1}^{(k)}&a_{m1}^{(k)}&\cdots&a_{mn}^{(k)}\\
    \end{bmatrix},\ k=1,2,\cdots\]
\end{defn}
\subsubsection{矩阵序列的极限}
与向量序列的极限类似,矩阵序列的极限定义如下:
\begin{defn}{}{}
    若\[\lim\limits_{k\to\infty}a^{(k)}_{ij}=a_{ij}\]则\[\lim\limits_{k\to\infty}\A^{(k)}=\A\]
\end{defn}

\subsubsection{矩阵序列极限的性质}
\begin{them}{}{}
    设$\lim\limits_{k\to\infty}\A^{(k)=\A}, \lim\limits_{k\to\infty}\B^{(k)=\B, \alpha, \beta\in\C}$则\begin{enumerate}
        \item $\lim\limits_{k\to\infty}(\alpha\A^{(k)}+\beta\B^{(k)})=\alpha\A+\beta\B$
        \item $\lim\limits_{k\to\infty}\A^{(k)}\B^{(k)}=\A\B$
        \item 当$\A^(k)$与$\A$都可逆时,$\lim\limits_{k\to\infty}(\A^{(k)})^{-1}=\A^{-1}$
    \end{enumerate}
\end{them}
下面的定理给出矩阵序列收敛条件:
\begin{them}{}{5.1.2}
    设$\Vert\bs{\cdot}\Vert$是$\C^{m\times n}$上任一矩阵范数,$\C^{m\times n}$中矩阵序列收敛于$\A$的充要条件是\[\lim\limits_{k\to\infty}\Vert\A^{(k)}-\A\Vert=0\]
\end{them}
与向量序列的收敛判断类似,定理\ref{them:5.1.2}指出了矩阵序列收敛条件,下面的定义将会指出何为收敛矩阵。
\begin{defn}{}{}
    设$\A\in\C^{n\times n}$,若$\lim\limits_{k\to\infty}\A^k=0(k\text{为正整数})$,则称$\A$为\textbf{收敛矩阵}.
\end{defn}
下面举一个例子,来表示何为收敛矩阵
\begin{example}
\[\A^{(k)}=\begin{bmatrix}
    \frac{1}{2^k}&0\\
    0&\frac{1}{2^k}
\end{bmatrix}\]
我们知道\[\lim\limits_{k\to\infty}\frac{1}{2^k}=0\]自然就可以推出\[\lim\limits_{k\to\infty}\A^k=0\]就代表矩阵$\A$是一个收敛矩阵。
\end{example}

下面介绍一下矩阵能够成为收敛矩阵的条件
\begin{them}{}{}
    设$\A\in\C^{n\times n}$,则$\A$为收敛矩阵的充要条件是谱半径$r(\A)<1$
\end{them}
证明如下:

\begin{proof}
    
    充分性证明:

    由$\A\in\C^{n\times n}$,再由Jordan标准型的知识可知,存在矩阵$\P$,使得\[\P^{-1}\A\P=\J=diag(J_{r_1}(\lambda_1),J_{r_2}(\lambda_2),\cdots,J_{r_s}(\lambda_s) )\]
    
    由矩阵乘法的知识,不难看出\[\A^k=\P\J^k\P^{-1}\]

    而\[\J_{r_i}^k(\lambda_i)=\begin{bmatrix}
        f_k(\lambda_i)&f'_k(\lambda_i)&\cdots&\frac{f^{(r_i-1)}_k(\lambda_i)}{(r_i-1)!}\\
        &f_k(\lambda_i)&\cdots&\frac{f^{(r_i-2)}_k(\lambda_i)}{(r_i-2)!}\\
        &&\ddots&\vdots\\
        &&&f_k(\lambda_i)
    \end{bmatrix}\]

    其中,$f_k(\lambda_i)=\lambda_i^k$

    由于$r(\A)<1$,则可以知道矩阵$\A$的所有特征值均小于1,那不难看出,当$k\to\infty$时,$f_k(\lambda_i)=\lambda_i^k\to0$, 从而$f_k^{(l)}(\lambda_i)\to0, \J_{r_i}^k(\lambda_i)\to0$

    因此,$\J\to0$,故$\A\to0$,从而知道$\A$是一个收敛矩阵,充分性证毕。

    必要性证明:

    由于$\A^k\to0$,故$\J_{r_i}^k(\lambda_i)\to0$, 故$\lambda_i^k\to 0$,从而可以知道$\A$的所有特征值绝对值必定小于1,因此$r(\A)<1$,必要性证毕。
\end{proof}
\subsection{矩阵级数}
介绍完矩阵序列,接下来介绍矩阵级数。

\begin{defn}{}{}
    设$\A^{(k)}$是$\C^{m\times n}$的矩阵序列,称\[\sum_{k=1}^{\infty}\A^{(k)}=\A^{(1)}+\A^{(2)}+\cdots+\A^{(k)}+\cdots\]为\textbf{矩阵级数},称$\bs{S}^{(N)}=\sum\limits_{k=1}^{N}\A^{(k)}$为矩阵级数的\textbf{部分和},如果$\lim\limits_{N\to\infty}\bs{S}^{(N)}=\bs{S}$,则称$\sum\limits_{k=1}^{\infty}\A^{(k)}$\textbf{收敛}
\end{defn}

上面的定义是从矩阵序列的角度定义的矩阵级数,接下来,与研究数项级数类似,如何判断矩阵级数绝对收敛?
\begin{defn}{}{}
    如果$mn$个数项级数
    \[\sum_{k=1}^{\infty}a_{ij}^{(k)},i=1,2,\cdots,m;j=1,2,\cdots,n\]都绝对收敛,则称矩阵级数$\sum_{k=1}^{infty}\A^{(k)}$绝对收敛。
\end{defn}

上面的定义是通过数项级数绝对收敛推出矩阵级数的绝对收敛,还可以通过利用范数来确定矩阵级数是否绝对收敛:
\begin{them}{}{}
    在$\C^{n\times n}$中,$\sum\limits_{k=1}^{\infty}\A^{(k)}$绝对收敛的充要条件是正项级数$\sum\limits_{k=1}^{\infty}\Vert\A^{(k)}\Vert$收敛。
\end{them}

证明如下:

\begin{proof}

    若$\sum\limits_{k=1}^{\infty}\A^{(k)}$绝对收敛,则$\sum\limits_{k=1}^{\infty}|a_{ij}^{(k)}\leq M$

    因此,根据矩阵1-范数,有\[\sum_{k=1}^{N}\Vert\A^{(k)}\Vert_{m_1}=\sum_{k=1}^{N}\left(\sum_{i=1}^{m}\sum_{j=1}^{n}|a_{ij}^{(k)}|\right)\leq mnM\]

    因此,若矩阵1-范数收敛,则任意矩阵范数一定收敛(小收敛则大收敛)
若任意矩阵范数收敛,那必定矩阵1-范数收敛,由于一定存在\[|a_{ij}^{(k)}|\leq\Vert\A^{(k)}\Vert_{m_1}\]则可以得出$\sum_{k=1}^{\infty}a_{ij}^{(k)}$绝对收敛,证毕。
\end{proof}

\subsection{矩阵幂级数}
\begin{them}{}{Neumann定理}
    方阵$\A$的Neumann级数\[\sum_{k=0}^{\infty}\A^k=\I+\A+\A^2+\cdots+\A^k+\cdots\]收敛的充要条件是$r(\A)<1$,且收敛时,和为$(\I-\A)^{-1}$
\end{them}

证明过程略,感兴趣的可以看书。

在学习微积分的无穷级数内容中,我们知道,幂级数存在收敛半径,那矩阵幂级数的收敛半径要如何求呢?下面的定理给出了答案:
\begin{them}{}{}
    设幂级数\[f(\bs{z})=\sum_{k=0}^{\infty}c_k\bs{z}^k\]的收敛半径为$r$,如果方阵$\A$满足$r(\A)<r$,则矩阵幂级数绝对收敛,如果$r(\A)>r$,则矩阵幂级数发散。
\end{them}
\section{矩阵函数}
上一节是为本节内容作为铺垫的,本节将会利用矩阵级数来研究矩阵函数。

\subsection{函数与幂级数}
说到函数,可能大多数人都会想到函数都是类似于下面的形式:\[f(x)=x, f(x)=x^2,\cdots\]

可在最一开始说,我们将会用矩阵级数的内容来研究矩阵函数,这两者存在何种关系呢?

如果还记得当时学无穷级数——函数展开成幂级数和幂级数的和函数的相关知识的话,就会知道,函数和无穷级数是离不开的,比如,你可能还记得下面这个式子:\[\sin x=x-\frac{1}{3!}x^3+\frac{1}{5!}x^5-\frac{1}{7!}x^7+\cdots\]

这就建立起了一个函数与幂级数的关系,事实上,常见初等函数都可以展开成幂级数的形式,这里就不一一介绍了,具体请翻阅高等数学教材。

这种幂级数与函数之间的关系放到矩阵中依旧成立,这就是为什么说用矩阵幂级数的内容来研究矩阵函数的原因,下面给出矩阵函数的定义:
\begin{defn}{}{}
    设幂级数$\sum\limits_{k=0}^{\infty}c_k\bs{z}^k$的收敛半径为$r$,且当$\bs{z}<r$时,幂级数收敛域$f(\bs{z})$,即\[f(\bs{z})=\sum_{l=0}^{\infty}c_k\bs{z}^k,\ \ \ |\bs{z}|<r\]

    如果$\A\in\C^{n\times n}$满足$r(\A)<r$,则称收敛的矩阵幂级数$\sum\limits_{k=0}^{\infty}c_k\bs{z}^k$的和记为矩阵函数,记为$f(\A)$,即\[f(\A)=\sum_{k=0}^{\infty}c_k\A^k\]

    将$f(\A)$的方阵$\A$换成$\A\bs{t}, \bs{t}$为参数,就会得到\[f(\A\bs{t})=\sum_{k=0}^{\infty}c_k(\A\bs{t})^k\]
\end{defn}

定义太长可以不看,简单来说就是定义前面的介绍,与普通的数项函数与幂级数的概念类似。
\subsection{常见的矩阵函数}
下面介绍了一些常见的矩阵函数,\textbf{这些函数的展开式应熟稔于心} (每天起床头件事,先背一遍展开式 (手动狗头)):
\begin{enumerate}
    \item \[e^{\A}=\E+\A+\frac{1}{2!}\A^2+\cdots+\frac{1}{k!}\A^k+\cdots=\sum_{k=0}^{\infty}\frac{(-1)^k}{(2k+1)!}\A^{2k+1}\]
    \item \[\sin\A=\A-\frac{1}{3!}\A^3+\frac{1}{5!}\A^5-\frac{1}{7!}\A^7+\cdots=\sum_{k=0}^{\infty}\frac{(-1)^k}{(2k+1)!}\A^{(2k+1)}\]
    \item \[\cos\A=\E-\frac{1}{2!}\A^2+\frac{1}{4!}\A^4-\frac{1}{6!}\A^6+\cdots=\sum_{k=0}^{\infty}\frac{(-1)^k}{(2k)!}\A^{(2k)}\]
    \item \[(\bs{E}-\A)^{-1}=\bs{E}+\A+\A^2+\cdots+\A^k+\cdots=\sum_{k=0}^{\infty}\A^k\]
    \item \[\ln(\bs{E}+\A)=\A-\frac{1}{2}\A^2+\frac{1}{3}\A^3-\frac{1}{4}\A^4+\cdots=\sum_{k=0}^{\infty}\frac{(-1)^k}{k+1}\A^{k+1}\]
\end{enumerate}

\subsection{矩阵函数值的计算}
只有函数的形式还没有用,我们最终要像计算普通函数那样把矩阵函数的函数值给计算出来,有三种方法可以计算矩阵函数值。
\subsubsection{相似对角化计算矩阵函数值}
若矩阵$\A$可以相似对角化,则存在下面的等式\[\P^{-1}\A\P=\bs{\Lambda}=diag(\lambda_1, \lambda_2, \cdots,\lambda_n)=\D\]

故\[\begin{aligned}
    f(\A)&=\sum_{k=0}^{\infty}c_k\A^k=\sum_{k=0}^{\infty}c_k(\P\D\P^{-1})^k=\P\left(\sum_{k=0}^{\infty}c_k\D^k\right)\P^{-1}\\
    &=\P\begin{bmatrix}
        \sum\limits_{k=0}^{\infty}c_k\lambda_1^k&&\\
        &\ddots&\\
        &&\sum\limits_{k=0}^{\infty}c_k\lambda_n^k
    \end{bmatrix}\P^{-1}
    =\P\begin{bmatrix}
        f(\lambda_1)&&\\
        &\ddots&\\
        &&f(\lambda_n)
    \end{bmatrix}\P^{-1}
\end{aligned}\]

其中\[f(\lambda_i)=\sum_{k=0}^{\infty}c_k\lambda_i^k\]

因此,使用相似对角化的方法求解矩阵函数值的步骤为:
\begin{enumerate}[leftmargin=4em]
    \item 将矩阵进行相似对角化
    \item 直接代公式算答案
\end{enumerate}
\textbf{注意}:使用该方法前,请务必确认该矩阵是否可以相似对角化,有关相似对角化的内容,可以参考第零章相似理论的相关知识,或查阅第一章Jordan标准型的相关知识。

\subsubsection{使用Jordan标准型的方式计算矩阵函数值}
设\[\P^{-1}\A\P=\J=diag(\J_1, \J_2,\cdots,\J_s),\J_i=\begin{bmatrix}
    \lambda_1&1&&\\
    &\ddots&\ddots&\\
    &&\lambda_i&1\\
    &&&\lambda_i
\end{bmatrix}_{m_i\times m_i}\]

则\[f(\J_i)=\sum_{k=0}^{\infty}a_k\J_i^k=\begin{bmatrix}
    f(\lambda_i)&\frac{1}{1!}f'(\lambda_i)&\cdots&\frac{1}{(m_i-1)!}f^{(m_i-1)}(\lambda_i)\\
    &f(\lambda_i)&\cdots&\vdots\\
    &&\ddots&\frac{1}{1!}f'(\lambda_i)\\
    &&&f(\lambda_i)
\end{bmatrix}\]

故\[
\begin{aligned}
    f(\A)&=\sum_{k=0}^{\infty}a_k\P\J^k\P^{-1}=(\sum_{k=0}^{\infty}a_k\J^k)\P^{-1}\\
    &=\P\begin{bmatrix}
        \sum\limits_{k=0}^{\infty}a_k\J^k_1&&\\
        &\ddots&\\
        &&\sum\limits_{k=0}^{\infty}a_k\J^k_s
    \end{bmatrix}\P^{-1}=\P\begin{bmatrix}
        f(\J_1)&&\\
        &\ddots&\\
        &&f(\J_s)
    \end{bmatrix}\P^{-1}
\end{aligned}
\]

因此,使用Jordan标准型的方法求解矩阵函数值的步骤为:
\begin{enumerate}[leftmargin=4em]
    \item 求Jordan标准型
    \item 代公式计算
\end{enumerate}
\subsubsection{使用数项级数求和方式计算矩阵函数值}
第一种方法有局限性,第二种方法需要较大计算量,下面将要介绍的这种方法计算量不大,但是需要记住常见函数的矩阵函数展开式。

在开始说计算方法之前,首先介绍哈密尔顿-凯莱定理:
\begin{them}{哈密尔顿-凯莱定理}{}
    设$\A$是数域$\mathbb{P} $上的一个$n\times n$矩阵,$f(\lambda)=|\lambda\bs{E}-\bs{A}|$是$\A$的特征多项式,则\[f(\A)=\A^n-b_{n-1}\A^{n-1}-\cdots-b_1\A-b_0\bs{E}=0\]
\end{them}

这个定理有什么用呢?我们继续看。

由定理,我们可以得到下面的等式关系:
\[\A^n=b_{n-1}\A^{n-1}+\cdots+b_1\A+b_0\bs{E}\]

如果我们想要计算$\A^{n+1}$,那只需要计算\[\A^n\cdot\A\]的结果即可,即
\[
\begin{aligned}\A^{n+1}&=\A^n\A=b_0\A+b_1\A^2+\cdots+b_{n-1}\A^n
\end{aligned}\]
而\[\A^n=b_{n-1}\A^{n-1}+\cdots+b_1\A+b_0\bs{E}\]因此,我们重新将式子整理一下合并同类项就可以得到一个新的式子

由此,我们会发现,再计算更高次的矩阵乘法时,我们完全可以用低阶矩阵乘积的结果来表示出来,从而简化了计算,这就是该定理的作用。

可这对计算矩阵函数值有什么用呢?由矩阵幂级数的知识,一个矩阵函数可以写成\[f(\A)=c_0\bs{E}+c_1\A+\cdots+c_k\A^k+\cdots\]的形式,由哈密尔顿-凯莱定理,最高次项可以用低阶来表示,因此可以重新合并同类项,最终就会变成\[(c_0+c_nb_0+c_{n+1}b_0^{(1)}+\cdots)\bs{E}+(c_1+c_nb_1+c_{n+1}b_1^{(1)}+\cdots)\A+\cdots\]的形式,而\[(c_0+c_nb_0+c_{n+1}b_0^{(1)}+\cdots)\text{和}(c_1+c_nb_1+c_{n+1}b_1^{(1)}+\cdots)\]都是数项级数,因此可以使用数项级数的方式来求解矩阵函数函数值。

因此,使用求解数项级数的方法求解矩阵函数值的步骤为:
\begin{enumerate}[leftmargin=4em]
    \item 求特征方程$|\lambda\bs{E}-\bs{A}|$的值,利用哈密尔顿-凯莱定理进行简化
    \item 代公式计算
\end{enumerate}
(如果只看上面这些抽象方法肯定看不懂,因此每一个方法应该都要加一个例子的,但是时间有限并且这些例子书上都有,所以可以看书或者PPT的例子就不在这里写了,等有时间再写)

\subsection{矩阵函数的其他性质}
本节的最后,再介绍一些矩阵函数的其他性质:
\begin{enumerate}[leftmargin=4em]
    \item 如果$\A\B=\B\A$,则$e^{\A}e^{\B}=e^{\B}e^{\A}=e^{\A+\B}$
    \item 如果$\A\B=\B\A$,则\begin{enumerate}
        \item $\cos(\A+\B)=\cos\A\cos\B-\sin\A\sin\B$
        \item $\sin(\A+\B)=\sin\A\cos\B+\cos\A\sin\B$
    \end{enumerate}
\end{enumerate}

这些性质与代数上的性质都类似,只不过需要注意一下这些性质的成立是基于$\A\B=\B\A$的情况下才可以,其他的就没有什么需要注意的了。
\ifx\allfiles\undefined
\end{document}
\fi